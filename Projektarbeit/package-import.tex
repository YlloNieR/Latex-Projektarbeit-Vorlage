%!TEX root = rootOfProject.tex
\usepackage[utf8]{inputenc} 
% encoding für Umlaute
\usepackage[ngerman]{babel}
\usepackage{csquotes}
% ermöglicht deutsche Silbentrennung
\usepackage[T1]{fontenc} 
% Fontencoding Format ermöglicht Suche nach Worten mit Ö
\usepackage{graphicx} 
% um Grafiken mit einzubinden z.B. jpeg
\usepackage{lmodern}
% mehr Schriftarten
\usepackage{geometry}
% für Layout config
\usepackage[toc,section=section,acronym]{glossaries}
% erstellt Glossar 
\usepackage{silence}
\WarningFilter{scrreprt}{Usage of package `fancyhdr'}
% verhindert von fancyhdr Error
\usepackage{pdfpages}
% Zur Einbindung externer pdf Dateien
\usepackage[
	colorlinks,
	pdfpagelabels,
	pdfstartview = FitH,
	bookmarksopen = true,
	bookmarksnumbered = true,
	linkcolor = black,
	plainpages = false,
	hypertexnames = false,
	citecolor = black
]{hyperref}
% fügt Hyperlinks hinzu
\usepackage[printonlyused]{acronym}
% Zur Verwenung von Acronymen / Abkürzungen
\usepackage{fancyhdr}
% Für Header und Footer Style

% \usepackage{helvet}
% Schriftart= helvetica 
% \renewcommand{\familydefault}{\sfdefault}

\usepackage{uarial}
% Schriftart= arial
\renewcommand{\familydefault}{\sfdefault}
% setze default Schriftart
\usepackage{ragged2e}
% Schriftarten wie Blockschrift

\usepackage{xcolor}
% Für Tabellenfarben wie Projektphasen

\usepackage[
	backend=biber,
	citestyle=numeric-comp,
	sorting=none,
	sortcites=true,
	block=none
	]{biblatex}
\addbibresource{bibShow.bib}



