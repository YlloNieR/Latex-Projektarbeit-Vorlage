%!TEX root = rootOfProject.tex
% verwendete Variabeln:
\newcommand{\auditCommittee}{IHK Stadtname}

\newcommand{\mainTitle}{Dokumentation zur betrieblichen Projektarbeit}
\newcommand{\projectTitleLong}{ProjektarbeitnameLang}
\newcommand{\projectTitleShort}{ProjektarbeitnameKurz}

\newcommand{\companyNameLong}{FirmennameLang}
\newcommand{\companyNameShort}{FirmennameKurz}

\newcommand{\companyLocationStreet}{Strasse}
\newcommand{\companyLocationNumber}{Nummer}
\newcommand{\companyLocationPostcode}{PLZ}
\newcommand{\companyLocationCity}{Stadt}
\newcommand{\companyLocationState}{Bundesland}

\newcommand{\apprenticeship}{Fachinformatiker für ...}
\newcommand{\deliveryPlace}{IhkOrt}
\newcommand{\submissionDate}{20.09.2021}
\newcommand{\authorName}{Yllonier}
\newcommand{\dateOfCreation}{12.08.2021}
\newcommand{\logo}{img/sample-logo.png}

% in Metadaten verwendete Variabeln
\title{\mainTitle}
\author{\authorName}



